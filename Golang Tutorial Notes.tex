\documentclass[11pt, oneside]{article}   	% use "amsart" instead of "article" for AMSLaTeX format
\usepackage{geometry}                		% See geometry.pdf to learn the layout options. There are lots.
\geometry{letterpaper}                   		% ... or a4paper or a5paper or ...
%\geometry{landscape}                		% Activate for for rotated page geometry
%\usepackage[parfill]{parskip}    		% Activate to begin paragraphs with an empty line rather than an indent
\usepackage{graphicx}				% Use pdf, png, jpg, or eps§ with pdflatex; use eps in DVI mode
								% TeX will automatically convert eps --> pdf in pdflatex
\usepackage{amssymb} %Gives us check marks
\usepackage{amsmath} % For using text in equations
\usepackage{array}
\usepackage{textcomp} %gives proper ASCII uprigth single quotation marks and tilde
\usepackage[T1]{fontenc} % For proper braces in \texttt env
\usepackage[pdftex]{hyperref} %clickable links for table of contents
\hypersetup{
	colorlinks,
	linkcolor=black
} %sets up links to be black
\usepackage[normalem]{ulem} %For strike out /sout, use normalem to fix for \emph{} behaviour (changes to underline otherwise)

\newcolumntype{C}{>{\texttt\bgroup}c<{\egroup}}
\newcolumntype{L}[1]{#1\textwidth>{\texttt\bgroup}l{\egroup}}
\setlength{\parindent}{0pt}

\title{Golang Notes}
\author{Sabrina Jiang}
%\date{}							% Activate to display a given date or no date
%Keyboard shortcuts useful for this
	%C-l,c for changing [thing] to \[thing]
	%C-l,C-e for \emph{} [Can be used with selecting text first]
	%C-l,C-b for \textbf{}
	%C-l,C-u for \underline
	%C-l,C-t for \texttt{}
	%C-l,e for begining an environment
	%Tab escapes the environment
	%To go to a section C-r
	%To fold all code: Cmd+K, Cmd+1
	%To unfold current: Cmd+Opt+]
	%To unfold all: Cmd+K, Cmd+J

\begin{document}
\maketitle
\tableofcontents

\newpage

\section{Basics}
  \begin{itemize}
    \item Packages
      \begin{itemize}
        \item Programs start running in package \texttt{main}
        \item Can also import packages using the below syntax
          \begin{verbatim}
            import (
                "fmt"
                "math/rand"
            )
          \end{verbatim}
        \item Exported names are \textbf{capitalized} (e.g. \texttt{Pi} is exported from the package \texttt{math})
      \end{itemize}
    \item Functions
      \begin{itemize}
        \item Basic Function Syntax
          \begin{verbatim}
            func [functionName]([varOneName], [varTwoName] [varOneAndTwoType], [etc]) ([returnVarOne, returnVarTwo returnType, etc]) {
                return [thing here]
            }
          \end{verbatim}
          \begin{itemize}
            \item A return statement without arguments will return all named variables
          \end{itemize}
      \end{itemize}
    \item Variable Declaration
      \begin{itemize}
        \item Variables can be declared without a type (e.g. \texttt{var c})
        \item Variables that are initialised must have a type (e.g. \texttt{var i int = 2})
        \item Variables can also be declared with the \texttt{:=} shorthand (e.g. \texttt{k := 3})
          \begin{itemize}
            \item Variables declared this way have their type inferred
            \item e.g. \texttt{42} is an \texttt{int} while \texttt{3.142} is a \texttt{float64}
          \end{itemize}
        \item Constants cannot be declared with \texttt{:=}
        \item Variables declared with types but no values are initialized with zero values (\texttt{0} for numeric, \texttt{false} for boolean, \texttt{""} for strings)
        \item You can convert between types by using the type as a function (e.g. from \texttt{int} to \texttt{float64}, use \texttt{float64(i)})
      \end{itemize}
  \end{itemize}

\section{Flow Control}
  \begin{itemize}
    \item For Loop Syntax
      \begin{verbatim}
      for [initializer] ; [condition] ; [post statement] {
          [code here]
      }
      \end{verbatim}
      \begin{itemize}
        \item Note the lack of parentheses around the components of the for loop
        \item The init and post statements are optional (basically making this into a while loop)
        \item A for loop without a post statement is an infinite loop
      \end{itemize}
    \item If Syntax
      \begin{verbatim}
      if [statement] {
          [code]
      } else {
          [more code]
      }
      \end{verbatim}
    \item Switch Statement Syntax
      \begin{verbatim}
      switch [to be checked against] {
          case [case1]:
              [code execution]
          case [case2]:
              [code execution]
          default:
              [code default]
      }
      \end{verbatim}
      \begin{itemize}
        \item Once the code hits a case that succeeds it automatically breaks
        \item A switch statement without a init is defaulted to be checked against \texttt{true}
      \end{itemize}
    \item Defer
      \begin{itemize}
        \item Arguments are evaluated immediately but the function is not called until after
        \item Deferred functions are pushed onto a stack and executed in a \textbf{last-in-first-out}
      \end{itemize}
  \end{itemize}

\section{More Data Types}
  \begin{itemize}
    \item Pointers
      \begin{itemize}
        \item A type \texttt{*T} is a pointer to the value of \texttt{T}
        \item It's zero value is \texttt{nil}
        \item The \texttt{\&} operator generates a pointer to its operand
          \begin{verbatim}
            i := 42
            p = &i
          \end{verbatim}
      \end{itemize}
    \item Struct
      \begin{itemize}
        \item Collection of fields
        \item Constructed via the following
          \begin{verbatim}
          type [name] struct {
            [varName] [varType]
            [varName] [varType]
          }
          \end{verbatim}
        \item You can create a pointer to structs but do not need to dereference them in order to change values
      \end{itemize}
    \item Arrays
      \begin{itemize}
        \item Array declaration syntax
          \begin{verbatim}
          var a [10]int
          \end{verbatim}
        \item An array's length is part of it's type so you cannot change that
        \item To "change" array lengths, you need to use the \textbf{Slices}$^{\ref{Slices}}$ data type
      \end{itemize}
    \item Slices \label{Slices}
      \begin{itemize}
        \item Declared as \texttt{[]T}, e.g.
        \begin{verbatim}
          primes := [6]int{2, 3, 5, 7, 11, 13} // an array
          var s []int = primes[1:4] //a slice
        \end{verbatim}
        \item Slices are just a view into an array
        \item Changes to a slice will also change the underlying array
        \item You can create a slice without explicitly creating the referenced array
        \begin{verbatim}
          []bool{true, true, false}
        \end{verbatim}
          \begin{itemize}
            \item This creates an array with those values and then a slice that references said array
          \end{itemize}
        \item Slices can be created without explicitly stating the upper and lower bounds. The defaults are \texttt{0} and the highest bound of the referenced array
        \item Slices have both a \textbf{length} and \textbf{capacity}
          \begin{description}
            \item[length] is the number of elements the slice contains (obtained through \texttt{len(s)})
            \item[capacity] is the number of elements of the \emph{underlying} array (obtained through \texttt{cap(s)})
          \end{description}
        \item Nil slice
          \begin{itemize}
            \item A nil slice has length of 0, capacity of 0, and no underlying array
          \end{itemize}
        \item Creating a slice with \texttt{make}
          \begin{verbatim}
          [varName] := make([][varType], [varLength], [opt: varCap])
          \end{verbatim}
      \end{itemize}
  \end{itemize}

\end{document}
